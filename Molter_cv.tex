%%%%%%%%%%%%%%%%%%%%%%%%%%%%%%%%%%%%%%%%%
% Medium Length Graduate Curriculum Vitae
% LaTeX Template
% Version 1.1 (9/12/12)
%
% This template has been downloaded from:
% http://www.LaTeXTemplates.com
%
% Original author:
% Rensselaer Polytechnic Institute (http://www.rpi.edu/dept/arc/training/latex/resumes/)
%
% Important note:
% This template requires the res.cls file to be in the same directory as the
% .tex file. The res.cls file provides the resume style used for structuring the
% document.
%
% Modified by E. Molter (5/20/16)
%%%%%%%%%%%%%%%%%%%%%%%%%%%%%%%%%%%%%%%%%

%----------------------------------------------------------------------------------------
%	PACKAGES AND OTHER DOCUMENT CONFIGURATIONS
%----------------------------------------------------------------------------------------

\documentclass[margin, 10pt]{res} % Use the res.cls style, the font size can be changed to 11pt or 12pt here

\usepackage{helvet} % Default font is the helvetica postscript font
%\usepackage{newcent} % To change the default font to the new century
%schoolbook postscript font uncomment this line and comment the one
%above
\usepackage{hyperref}
\usepackage{amssymb}
\usepackage{etoolbox}
\usepackage{fancyhdr}
% new packages should be put in /opt/local/share/texmf-texlive/

\newtoggle{pubslist}
\toggletrue{pubslist}

\newcommand\tab[1][1cm]{\hspace*{#1}}

\begin{document}

%----------------------------------------------------------------------------------------
%	NAME AND ADDRESS SECTION
%----------------------------------------------------------------------------------------

\moveleft.5\hoffset\centerline{\large\bf Dr. Edward M. Molter} % Your name at the top
 
\moveleft\hoffset\vbox{\hrule width\resumewidth height 1pt}\smallskip % Horizontal line after name; adjust line thickness by changing the '1pt'
 
% \moveleft.5\hoffset\centerline{9808 53rd Ave., College Park, MD 20740} % Your address
\moveleft.5\hoffset\centerline{Nickname: Ned \, \footnotesize{$\blacklozenge$} \, \normalsize{pronouns: he/him}}
\moveleft.5\hoffset\centerline{Postdoctoral Scholar, Earth and Planetary Science Department, UC Berkeley}
\moveleft.5\hoffset\centerline{emolter@berkeley.edu}
\moveleft.5\hoffset\centerline{(414) 573-2014}
\moveleft.5\hoffset\centerline{\url{https://emolter.github.io/}}

%----------------------------------------------------------------------------------------

\begin{resume}
	
%----------------------------------------------------------------------------------------
%	RESEARCH INTERESTS
%----------------------------------------------------------------------------------------

\section{RESEARCH INTERESTS}
Planetary atmospheres; planetary rings; radiative transfer; atmospheric dynamics; extreme weather; climate change; photochemistry; interferometry, astronomical software development
 
%----------------------------------------------------------------------------------------
%	EDUCATION SECTION
%----------------------------------------------------------------------------------------

\section{EDUCATION}

{\bf Ph.D. Astrophysics}, University of California, Berkeley \hfill
{\bf August 2022} \\
\tab {\it Thesis:} ``Cloud Formation and Circulation in Planetary Tropospheres %\\ \tab \tab \tab 
from Remote-Sensing Data'' \\
\tab {\it Advisers:} Dr. William Collins, Dr. Imke de Pater \\
{\bf M.A. Astrophysics}, University of California, Berkeley \hfill
{\bf December 2018} \\
{\bf B.A. Physics, Summa Cum Laude}, Macalester College \hfill {\bf
  May 2015} \\
%\tab {\it Emphasis:} Astronomy \\
\tab {\it Thesis:}  ``Constraining the Properties of the Metal-Poor
ISM with \\ \tab \tab \tab Interferometric CO Observations of Low Metallicity Dwarf
Galaxies'' \\
\tab {\it Adviser:} Dr. John Cannon

%----------------------------------------------------------------------------------------
%	RESEARCH POSITIONS
%----------------------------------------------------------------------------------------

\section{RESEARCH POSITIONS}

{\bf CIPS Postdoctoral Scholar}, Dept. of Earth \& Planetary Science, UC Berkeley \hfill {\bf
  Sep 2022 - present} \\
{\bf Graduate Student Researcher}, Lawrence Berkeley National Lab \hfill {\bf
  Aug 2019 - Aug 2022} \\
 \tab {\it Adviser:} Dr. William Collins \\
{\bf Graduate Student Researcher}, Dept. of Astronomy, UC Berkeley \hfill {\bf
  Jan 2017 - Aug 2019} \\
%\tab {\it Project Title:} Unknown \\
\tab {\it Adviser:} Dr. Imke de Pater \\
{\bf Visiting Scholar}, Keck Observatory \hfill {\bf
  Summer 2017} \\
%\tab {\it Project Title:} Cadenced Twilight Observations of Solar System Bodies \\
\tab {\it Adviser:} Dr. Carlos Alvarez \\
{\bf Research Assistant}, NASA Goddard Space Flight Center \hfill {\bf
  Aug 2015 - July 2016} \\
%\tab {\it Project Title:} Constraining Titan's Atmospheric Photochemistry using ALMA \\
\tab {\it Adviser:} Dr. Conor Nixon \\
{\bf Undegraduate Research Assistant}, Macalester College \hfill {\bf Sep
2014 - May 2015} \\
\tab {\it Adviser:} Dr. John Cannon \\
{\bf NSF REU Research Student}, US Geological Survey/Northern Arizona U. \hfill {\bf Summer 2014} \\
%\tab {\it Project Title:} Characterization of Hollows on Mercury with
%  MESSENGER \\
\tab {\it Adviser:} Dr. Colin Dundas \\
{\bf Visiting Research Student}, Universit\'e Libre de Bruxelles
\hfill {\bf Fall 2013} \\
%\tab {\it Project Title:} Theoretical Mass Limits on Magnetized White
%  Dwarfs \\
\tab {\it Adviser:} Dr. Nicolas Chamel \\
{\bf Undergraduate Research Assistant}, Macalester College \hfill
{\bf Summer 2013} \\
%\tab {\it Project Title:} CARMA CO Observations of the Extremely
%  Metal Poor Dwarf Galaxy Leo P \\
\tab {\it Adviser:} Dr. John Cannon \\



%----------------------------------------------------------------------------------------
%      FUNDING -UPDATE
%----------------------------------------------------------------------------------------

%\section{PI GRANT FUNDING}


%----------------------------------------------------------------------------------------
%      PUBLICATIONS
%----------------------------------------------------------------------------------------

\iftoggle{pubslist}{

\section{REFERREED JOURNAL ARTICLES}

\textit{*Student Advised} \hfill \url{https://orcid.org/0000-0003-3799-9033} \\
\begin{enumerate}
	
\item[21.] {\bf Molter, E. M.}, de Pater, I., Moeckel, C., ``Keck Near-Infrared Detections of Mab and Perdita'', Icarus Letters, in review
	
\item[20.]* Chavez, E., de Pater, I., Redwing, E., {\bf Molter, E. M.}, Roman, M. T., Zorzi, A., Alvarez, C., Campbell, R., de Kleer, K., Hueso, R., et al. ``Evolution of Neptune at Near-Infrared Wavelengths from 1994 through 2022'', Icarus, in review

\item[19.]* Chavez, E., Redwing, E., de Pater, I., Hueso, R., {\bf Molter, E. M.}, Wong, M. H.,  Alvarez, C., Campbell, R., de Kleer, K., et al.,  ``Drift Rates of Major Neptunian Features between 2018 and 2021'', Icarus, in press
	
\item[18.] de Pater, I., {\bf Molter, E. M.}, Moeckel, C. M. ``A Review of Radio Observations of the Giant Planets: Probing the Composition, Structure, and Dynamics of Their Deep Atmospheres'', Remote Sensing, 15, 5, 1313 (2023) \url{https://doi.org/10.3390/rs15051313}
	
\item[17.] Zhang, L., Risser, M., {\bf Molter, E. M.}, Wehner, M. F., O'Brien, T. A., ``Accounting for the spatial structure of weather systems in detected changes in precipitation extremes'', Weather \& Climate Extremes, 100499 (2022) \url{https://doi.org/10.1016/j.wace.2022.100499}

\item[16.] {\bf Molter, E. M.}, Collins, W. D., Risser, M. D., ``Quantitative Precipitation Estimation of Extremes in CONUS with Radar Data'', Geophysical Research Letters, 48, 16 (2021) \url{https://doi.org/10.1029/2021GL094697}

\item[15.] Villanueva, G., Cordiner, M., Irwin, P., et al., incl. {\bf Molter, E.}, ``No evidence of phosphine in the atmosphere of Venus from independent analyses'', Nature Astronomy 5, 631-635 (2021) \url{https://doi.org/10.1038/s41550-021-01422-z}

\item[14.]* Zorzi, A., {\bf Molter, E. M.}, de Pater, I., Luszcz-Cook, S. H., Tollefson, J., Wong, M. H., ``Evolution of Neptune's Troposphere in 1994-2018 based on HST Observations'', Astronomy \& Astrophysics, in review 

\item[13.] Tollefson, J., de Pater, I., {\bf Molter, E. M.}, Sault, R. J., Butler, B. J., Luszcz-Cook, S., DeBoer, D., ``Neptune's Spatial Brightness Temperature Variations from the VLA and ALMA'', Planetary Science Journal 2, 3 (2021) \url{https://doi.org/10.3847/PSJ/abf837}

\item[12.] {\bf Molter, E. M.}, de Pater, I., Luszcz-Cook, S., Tollefson, J., Sault, R. J., Butler, B., de Boer, D., ``Tropospheric Composition and Circulation of Uranus with ALMA and the VLA'', Planetary Science Journal, 2, 1 (2021) \url{https://doi.org/10.3847/PSJ/abc48a}

\item[11.] Nixon, C. A., Thelen, A. E., Cordiner, M. A., Kisiel, Z., Charnley, S. B., {\bf Molter, E. M.}, Serigano, J., Irwin, P. G. J., Teanby, N., Kuan, Y., ``Detection of Cyclopropenylidene on Titan with ALMA'', Astronomical Journal, 160, 5 (2020) \url{https://doi.org/10.3847/1538-3881/abb679}

\item[10.] {\bf Molter, E. M.}, de Pater, I., Roman, M. T., Fletcher, L. N., ``Thermal Emission from the Uranian Ring System'', Astronomical Journal, 158, 47 (2019) \url{https://doi.org/10.3847/1538-3881/ab258c}

\item[9.] de Kleer, K., de Pater, I., {\bf Molter, E. M.}, Banks, E., Davies, A. G., Alvarez, C., Campbell, R., et al., ``Io's Volcanic Activity from Time Domain Adaptive Optics Observations: 2013-2018'', Astronomical Journal, 158, 29 (2019) \url{https://doi.org/10.3847/1538-3881/ab2380}

\item[8.] {\bf Molter, E. M.}, de Pater, I., Luszcz-Cook, S., Hueso, R., Tollefson, J., Alvarez, C., S\`anchez-Lavega, A., Wong, M. H., Hsu, A. I., Sromovsky, L. A., Fry, P. M., Delcroix, M., Campbell, R., de Kleer, K., Gates, E., Lynam, P. D., et al., ``Analysis of Neptune's 2017 Bright Equatorial Storm'', Icarus, 321, 324 (2019) \url{https://doi.org/10.1016/j.icarus.2018.11.018}

\item[7.] Thelen, A. E., Nixon, C. A., Chanover, N. J., Cordiner, M. A., {\bf Molter, E. M.}, Teanby, N. A., Irwin, P. G. J., Serigano, J., Charnley, S. B., ``Abundance Measurements of Titan's Stratospheric HCN, HC$_3$N, C$_3$H$_4$, and CH$_3$CN from ALMA observations'', Icarus, 319, 417 (2019) \url{https://doi.org/10.1016/j.icarus.2018.09.023}

\item[6.] Cordiner, M. A., Nixon, C. A., Charnley, S. B., Teanby, N. A., {\bf Molter, E. M.}, Kisiel, Z., Vuitton, V., ``Interferometric Imaging of Titan's HC$_3$N, H$^{13}$\textrm{CCC}N, and HCCC$^{15}$N'', Astrophysical Journal Letters, 859, L15 (2018) \url{https://doi.org/10.3847/2041-8213/aac38d}

\item[5.] Thelen, A. E.,  Nixon, C. A., Chanover, N. J., {\bf Molter, E. M.}, Cordiner, M. A., Achterberg, R. K., Serigano, J., Irwin, P. G. J., Teanby, N., Charnley, S. B., ``Spatial variations in Titan's atmospheric temperature: ALMA and Cassini comparisons from 2012 to 2015'', Icarus, 307, 380 (2018) \url{https://doi.org/10.1016/j.icarus.2017.10.042}

\item[4.] Lai, J. C.-Y., Cordiner, M. A., Nixon, C. A., Achterberg, R. K., {\bf Molter, E. M.}, Teanby, N. A., Palmer, M. Y., Charnley, S. B., Lindberg, J. E., Kisiel, Z., Mumma, M. J., Irwin, P. G. J., ``Mapping Vinyl Cyanide and Other Nitriles in Titan’s Atmosphere Using ALMA'', Astronomical Journal, 154, 206 (2017) \url{https://doi.org/10.3847/1538-3881/aa8eef}

\item[3.] {\bf Molter, E. M.}, Nixon, C. A., Cordiner, M. A., Serigano, J., Irwin, P. G. J., Teanby, N. A., Charnley, S. B., Lindberg, J. E., ``ALMA Observations of
  HCN and its Isotopologues on Titan'', Astronomical Journal, 152, 2 (2016) \url{https://doi.org/10.3847/0004-6256/152/2/42}
  
\item[2.] Warren, S. R., {\bf Molter, E. M.}, Cannon, J. M., Bolatto, A. D., Adams, E. A. K., Bernstein-Cooper, E. Z., Giovanelli, R., Haynes, M. P., Herrera-Camus, R., Jameson, K., McQuinn, K. B. W., Rhode, K. L., Salzer, J. J., Skillman, E. D., ``CARMA
  CO Observations of Three Extremely Metal-Poor, Star-Forming
  Galaxies", Astrophysical Journal, 814, 30 (2015) \url{https://doi.org/10.1088/0004-637X/814/1/30}
  
\item[1.] Chamel, N., {\bf Molter, E.}, Fantina, A. F., Arteaga, D. P., ``Maximum strength of the magnetic field in the core of the most massive white dwarfs," Physical Review Letters D, 90, 043002 (2014) \url{https://doi.org/10.1103/PhysRevD.90.043002}
\end{enumerate}

}{}

\section{TELESCOPE TIME AWARDED}

%Improve format for this section!

\textbf{Atacama Large (sub-)Millimeter Array (ALMA)} 
\begin{enumerate}
	\item[2.] Primary Investigator, {\it Thermal Properties of the Uranian Rings}, 8.5 hours %ID: 2021.1.01017.S
	\item[1.] Primary Investigator, {\it Opacity Variability in Uranus's Troposphere}, 3.7 hours \\ Funding awarded (\$17,500) via NRAO Student Observing Support Award %ID: 2017.1.00855.S
\end{enumerate}
\vspace{-0.3cm}
\textbf{James Webb Space Telescope (JWST)} 
\begin{enumerate}
	\item[1.] co-Investigator, {\it ERS observations of the Jovian System as a demonstration of JWST's capabilities for Solar System science}, Instruments: Multiple; PIs: T. Fouchet and I. de Pater, 28.9 hours %ID: 1373
\end{enumerate}
\vspace{-0.3cm}
\textbf{W. M. Keck Observatory} 
\begin{enumerate}
	\item[3.] co-Investigator, {\it The Twilight Zone: Cadenced Twilight Observations of Solar System Bodies}, long-term program. Instruments: NIRC2, Osiris; PIs: I. de Pater, K. de Kleer, A. Davies, 2018-present. $>$80 activations, 0.5 hours each %33 Io, 8 Titan, 20 Uranus, 34 Neptune on website 
	\item[2.] co-Investigator, {\it Spatial Distribution of H$_2$S on Neptune and Uranus}, Instrument: OSIRIS; PI: I. de Pater, 1.0 nights %ID: 2019B\_U062
	\item[1.] co-Investigator, {\it Uranus from Equinox to Mid-Spring: Tropospheric Temperatures, Seasonal Changes, and Emerging Rings}, Instrument: Subaru COMICS; PI: J. Sinclair, 1.0 nights %ID: 2019B\_N059
\end{enumerate}
\vspace{-0.3cm}
\textbf{Very Large Array (VLA)}
\begin{enumerate}
	\item[1.] co-Investigator, {\it Seasonal Variations in the Microwave Emission of Uranus}, PI: Alex Akins, 18.0 hours %ID: VLA/21B-301
\end{enumerate}
\vspace{-0.3cm}
\textbf{Very Large Telescope (VLT)} 
\begin{enumerate}
	\item[2.] co-Investigator, {\it Uranus from Equinox to Mid-Spring: Temperature Structure, Photochemistry, Seasonal Changes, and Emerging Rings}, Instrument: VISIR; PI: M. Roman, 14.5 hours %ID: 0104.A-????
	\item[1.] co-Investigator, {\it Resolve Loki Patera on Jupiter’s Satellite Io with Matisse}, Instrument: MATISSE; PI: I. de Pater, 3 hours %ID: 0103.C-0089
\end{enumerate}
\vspace{-0.3cm}
\textbf{Paranal Observatory}
\begin{enumerate}
	\item[1.] co-Investigator, {\it Preparatory observations for GTO program on Matisse of Io's Loki Patera}, Instrument: NACO; PI: I. de Pater, XX activations, 0.5 hours each %ID: 0103.C-0088
\end{enumerate}
\vspace{-0.3cm}
\textbf{Lick Observatory} 
\begin{enumerate}
	\item[1.] Primary \& co-Investigator, {\it Origin \& Evolution of Storms, Clouds, and Hazes on Uranus and Neptune}, long-term program. Instrument: ShARCS; PIs: E. Molter, J. Tollefson, E. Redwing. $>$80 activations, 1 hour each %Ned: (2019B\_S011, 2019A\_S009), Josh: (2018B\_S000, 2018A\_S002, 2017B\_S000), Erin: ?
\end{enumerate}


%----------------------------------------------------------------------------------------
%   Open-source software
%----------------------------------------------------------------------------------------

\section{OPEN-SOURCE SOFTWARE}

I actively contribute to the open-source software ecosystem within planetary science:

\begin{itemize}

\item Contributed the \href{https://github.com/astropy/astroquery/tree/main/astroquery/solarsystem/pds}{Planetary Ring Node query tool} to the \texttt{astropy}-coordinated \texttt{astroquery} package

\item Developed the \href{https://www2.keck.hawaii.edu/inst/tda/TwilightZone.html#}{Twilight Zone} observing tools and public-facing website at Keck Observatory

\item Wrote the \href{https://github.com/emolter/nirc2_reduce}{\texttt{nirc2\_reduce} package} for processing Keck NIRC2 data, as published in e.g. \href{https://doi.org/10.1016/j.icarus.2018.11.018}{Molter et al. 2019}, \href{https://doi.org/10.3847/1538-3881/ab2380}{de Kleer et al. 2021}, Chavez et al. 2023a,b (in review)

\item Contributed Uranus \& Neptune cloud physics and MCMC support to the \texttt{radiobear} radiative transfer code, as published in e.g. \href{https://doi.org/10.3847/PSJ/abf837}{Tollefson et al. 2021}, \href{https://doi.org/10.3847/PSJ/abc48a}{Molter et al. 2021}, \href{https://doi.org/10.3390/rs15051313}{de Pater et al. 2023}

\item Co-maintainer of the \texttt{sunbear} radiative transfer code, as published in e.g. \href{https://doi.org/10.1016/j.icarus.2016.04.032}{Luszcz-Cook et al. 2016}, \href{https://doi.org/10.1016/j.icarus.2018.11.018}{Molter et al. 2019}, Zorzi et al. 2023 (in review). First public release coming soon!

\end{itemize}

I recently adopted the \texttt{showyourwork!} workflow for open and reproducible scientific publications; my first paper using this package is available \href{https://github.com/emolter/mab}{at this link}.


%----------------------------------------------------------------------------------------
%      TEACHING - UPDATE ME!!
%----------------------------------------------------------------------------------------

\section{TEACHING, OUTREACH, \& MENTORSHIP}

{\bf Mentor}, Berkeley Undergraduate Research Apprentice Program \hfill Spring 2022 - Present \\
{\bf Volunteer Organizer}, Berkeley Climate \& Impacts Research Hub \hfill Fall 2020 - Spring 2022 \\
{\bf Graduate Student Instructor}, UC Berkeley \\
\tab C162 Planetary Astrophysics \hfill Fall 2018 \\
\tab C12 The Planets \hfill  Spring 2017 \\
\tab C10 Introduction to General Astronomy \hfill Fall 2016 \\
{\bf Volunteer Panelist}, Branson School Science Symposium \hfill 2018, 2019 \\
{\bf Volunteer Instructor}, Splash @ Berkeley \hfill 2018 \\
{\bf Peer Mentor}, Berkeley Astronomy Dept.
\hfill Fall 2018 - Present \\
{\bf Orientation Leader}, Macalester College Dept of Student Affairs
\hfill Fall 2012 \\
{\bf Program Staff (full-time)}, Camp Becket/Chimney Corners YMCA,
Becket, MA
%. Designed curriculum for and led outdoor education classes for middle school students.
\hfill Summer 2012


%----------------------------------------------------------------------------------------
%      WORKSHOPS
%----------------------------------------------------------------------------------------

\section{PROFESSIONAL DEVELOMENT}

{\bf Astronomical Software Development Workshop}, Flatiron Institute, NY \hfill May 2022 \\ 
{\bf Graduate Climate Conference (GCC)}, Virtual \hfill October 2021 \\ 
{\bf Unlearning Racism in the Geosciences (URGE)} Berkeley Chapter, Virtual \hfill Fall 2020 \\ 
{\bf JPL Center for Climate Sciences Summer School}, Virtual \hfill August 2020 \\ 
{\bf Physics in Machine Learning Workshop}, Berkeley, California \hfill May 2019 \\
{\bf Very Large Array (VLA) Synthesis Imaging Workshop}, Socorro, New Mexico \hfill May 2018 \\
{\bf Very Large Array (VLA) Data Reduction Workshop}, Socorro, New Mexico \hfill October 2017 \\
{\bf JWST Early Release Science Proposal Writing Workshop}, Leiden, Netherlands \hfill May 2017 \\
{\bf Titan Aeronomy and Climate Workshop}, Reims, France \hfill June 2016 \\
{\bf Combined Array for Research in Millimeter Astronomy (CARMA)} \hfill August 2014 \\
  \tab {\bf Summer School}, Big Pine, CA  \\
{\bf Undergraduate ALFALFA Team Workshop}, Arecibo, Puerto Rico \hfill January 2014


%----------------------------------------------------------------------------------------
%      SEMINARS
%----------------------------------------------------------------------------------------

% \section{SEMINARS \& Colloquia}

% add invited at NRAO, Aug 31 2022

% {\bf UC Berkeley}, Department of Astronomy CIPS Seminar
% \hfill Oct 2018 \\
% {\bf UC Berkeley}, Department of Astronomy CIPS Seminar
% \hfill Dec 2017 \\
% {\bf UC Berkeley}, Department of Astronomy Lunch Colloquium
% \hfill Aug 2016 \\
% {\bf Macalester College}, Physics \& Astronomy Department Colloquium
% \hfill Apr 2015 \\
% {\bf Macalester College}, PHYS 113 (Modern Astronomy) guest lecture
% \hfill Nov 2014 \\
% {\bf Undergraduate ALFALFA Team Workshop} \hfill Jan 2014 \\



%----------------------------------------------------------------------------------------
%      POSTERS SECTION
%----------------------------------------------------------------------------------------

\section{CONFERENCE PRESENTATIONS}

%ADD DOIs HERE

\begin{enumerate}
	
\item[15.] ``The Atmosphere and Rings of Uranus at 25 mas Resolution with ALMA'', AGU Fall Meeting, P23B-07 (2022)
\item[14.] ``A Storm-Resolving Data Set for Analysis of Precipitation at its Native Scale, Diagnosis of Cloud-Resolving Models, and Development of Next-Generation Parameterizations'', AGU Fall Meeting, A45Q-2082 (2021)
\item[13.] ``Quantitative Precipitation Estimation of Extremes over the Continental United States with Radar Data'', AMS Annual Meeting, 2A.1 (2021) \href{https://ams.confex.com/ams/101ANNUAL/meetingapp.cgi/Paper/381114}{[\texttt{click for video recording}]}
\item[12.] \textbf{(Invited)} ``Thermal Measurements of the Ring System of Uranus'', AGU Fall Meeting, P017-03 (2020)
\item[11.] ``Quantitative Precipitation Estimation of Extremes over the Continental United States with Radar Data'', AGU Fall Meeting, A042-0014 (2020)
%\item[11.] RGMA PI meeting
\item[10.] ``Uranus's Tropospheric Circulation and Composition with ALMA and the VLA'', EPSC/DPS Meeting 13, 726-1 (2019)
\item[9.] ``Uranian Atmosphere and Rings Probed with ALMA Observations'', AAS/DPS Meeting, 50, 104.07 (2018)
\item[8.] ``Mapping circulation and chemistry in Uranus's deep atmosphere with radio observations'', Astrophysical Frontiers in the Next Decade and Beyond Meeting (2018)
\item[7.] ``Discovery of a Bright Equatorial Storm on Neptune'', AGU Fall Meeting, P31D-2856 (2017)
\item[6.] ``Isotopic Ratios in Nitrile Species on Titan using ALMA", Titan Aeronomy \& Climate Workshop, \#37 (2016)
\item[5.] ``Observations of HCN and its Isotopologues on Titan using ALMA", AAS, 227, \#141.19 (2016)
\item[4.] ``Vertical Profiles and Isotopic Ratios in HCN and its Isotopologues from ALMA Observations of Titan", AAS/DPS, 47, \#310.15 (2015)
\item[3.] ``Testing for the Influence of Insolation on Formation and Growth of Hollows on Mercury," LPSC, 46, \#1489 (2015)
\item[2.] ``CO Observations of DDO 68: An Extreme Outlier on the Mass-Metallicity Relation", AAS, 225, \#248.18 (2015)
\item[1.] ``The Low CO Luminosity of Three Extremely Metal-Poor Star-Forming Galaxies", AAS, 223, \#246.52 (2014)

\end{enumerate}


%----------------------------------------------------------------------------------------
%      AWARDS SECTION
%----------------------------------------------------------------------------------------

%\section{HONORS \& AWARDS}
%
%NRAO Student Observing Support Award (\$17,500) \hfill October 2018 \\ %$17,500
%Keck Visiting Scholar \hfill June 2017 \\
%Dr. Sherman W. Schultz Memorial Award \hfill May 2015 \\
%Phi Beta Kappa Honors Society Induction \hfill May 2015 \\
%National Merit Scholar \hfill 2011 \\
%Presidential Scholars Program Semifinalist \hfill May 2011 \\


%----------------------------------------------------------------------------------------
%      Publicity
%----------------------------------------------------------------------------------------

% MAKE THESE LINKED URLs!

\section{PUBLICITY}
% NRAO ngVLA eNews highlight
\href{https://news.berkeley.edu/2022/01/25/berkeley-astronomers-to-put-new-space-telescope-through-its-paces/}{{\bf Press Release}}, UC Berkeley, ``Berkeley Astronomers to Put New Space \hfill January 2022 \\Telescope Through its Paces'' \\
	%https://phys.org/news/2022-01-astronomers-james-webb-space-telescope.html
\href{https://futurism.com/scientists-gross-uranus-jokes}{{\bf Interview}}, Futurism, ``Here's What Uranus Scientists Think About Your  \hfill November 2021 \\
\tab Disgusting Jokes''  \\
\href{https://www.nasa.gov/feature/goddard/2020/nasa-scientists-discover-a-weird-molecule-in-titan-s-atmosphere}{{\bf Press Release}}, NASA, ``NASA Scientists Discover `Weird' Molecule \hfill October 2020 \\
\tab in Titan's Atmosphere''  \\
\href{https://news.berkeley.edu/2019/06/20/astronomers-see-warm-glow-of-uranuss-rings/}{{\bf Press Release}}, UC Berkeley, ``Astronomers see `warm' glow of Uranus's rings'' \hfill June 2019 \\ 
\href{https://www.nature.com/articles/d41586-018-07622-4}{{\bf Nature Research Highlight}}, ``Epic storm roils a tranquil region of Neptune'' \hfill December 2018 \\
\href{http://www.keckobservatory.org/new_storm_makes_surprise_appearance_on_neptune/}{{\bf Press Release}}, UC Berkeley/Keck Observatory, ``New Storm Makes Surprise \hfill August 2018 \\
	\tab Appearance on Neptune'' \\ %https://news.berkeley.edu/story_jump/twilight-observations-reveal-huge-storm-on-neptune/

%----------------------------------------------------------------------------------------
%      Advising
%----------------------------------------------------------------------------------------

%\section{ADVISING}
%Andrea Zorzi, Masters' Student, TU Delft \hfill 2018  \\



%----------------------------------------------------------------------------------------
%   PROFESSIONAL SERVICE -UPDATE ME- put mentoring and refereeing here
%----------------------------------------------------------------------------------------


%----------------------------------------------------------------------------------------
%   SELECTED COURSEWORK - good if I'm applying to earth science
%----------------------------------------------------------------------------------------

\section{SELECTED GRADUATE COURSEWORK}

I took advantage of the rich academic program at UC Berkeley by enrolling in classes throughout my graduate career, going well beyond the coursework requirements:

\begin{enumerate}
	
	\item[11.] Effective Mentoring in Higher Education \hfill Spring 2022
	
	\item[10.] Python Computing for Data Science \hfill Spring 2022
	
	\item[9.] Unlearning Racism in the Geosciences (URGE) \hfill Spring 2021
	
	\item[8.] Global Circulation of Planetary Atmospheres \hfill Fall 2020
	
	\item[7.] Computational Fluid Dynamics \hfill Fall 2020
	
	\item[6.] Atmospheric Physics and Dynamics (audit) \hfill Fall 2019
	
	\item[5.] Astrophysical Fluid Dynamics \hfill Spring 2018
	
	\item[4.] Radiation and its Interactions with Climate \hfill Fall 2017
	
	\item[3.] Solar System Astrophysics \hfill Fall 2017
	
	\item[2.] Astrophysical Techniques \hfill Spring 2017
	
	\item[1.] Radiative Processes in Astronomy \hfill Fall 2016
	
\end{enumerate}


%----------------------------------------------------------------------------------------
%	SKILLS SECTION  -UPDATE ME!!
%----------------------------------------------------------------------------------------

% \section{SOFTWARE}

%Data Reduction
%•	Common Astronomy Software Applications (CASA) – very experienced user
%o	ALMA flux calibration observations of Titan
%	Flux calibration observations for science
%	Custom reduction and imaging, i.e. editing pipeline steps
%	Spectral line data in Bands 3, 4, 6, and 7
%	Continuum and spectral line imaging
%o	ALMA observations of Uranus and its rings
%	Workflow development for subtracting arbitrary u-v plane models to reduce Gibbs ringing
%	Iterative self-calibration
%	Best practices for faint objects near bright sources
%o	VLA workshops
%	Synthesis Imaging Workshop, Socorro, New Mexico, May 2018
%	Data Reduction Workshop, Socorro, New Mexico, October 2017
%•	MIRIAD – experienced user, but it’s been a while
%o	Data reduction for observations of faint dwarf galaxies
%o	Combined Array for Research in Millimeter Astronomy (CARMA) summer school, August 2014
%•	Infrared imaging pipeline
%o	Developed a Python-based pipeline from scratch for calibration and imaging of infrared observations of planets
%o	Applied to Keck NIRC2 and OSIRIS instruments, as well as Lick ShARCS instrument
%o	https://github.com/emolter/nirc2_reduce 


% \section{SKILLS} 

% {\bf Computing} \\
% \tab {\it Languages:} Python, IDL, Unix \\
% \tab {\it Software:} NEMESIS, CASA, MIRIAD, LaTex, ArcMap,
% Mathematica \\
% \tab {\it Operating Systems:} Macintosh, Linux, Windows \\
% {\bf Observing} \\
% \tab CARMA, Arecibo (one week workshop using
% each) \\
% \tab Research grade 16" optical telescope, Macalester College
% (semester project) \\
% {\bf Languages} \\
% \tab French (conversational) \\
% \tab Dutch (beginner) \\

% %----------------------------------------------------------------------------------------
% %	OTHER EXPERIENCE SECTION
% %----------------------------------------------------------------------------------------

% \section{OTHER EXPERIENCE}

% {\sl Wilderness First Responder Certification} \hfill April 2018 \\
% NOLS, Flagstaff, AZ
% \begin{itemize}
% \item 80-hour hands-on course
% \end{itemize}

% {\sl Head Sound Technician} \hfill May 2014 - May 2015 \\
% Department of Music, Macalester College, Saint Paul, MN
% \begin{itemize}
% \item Scheduled and directed sound and tech workers for department concerts
% % \item Edit and distribute concert recordings using ProTools software
% \item Trained new sound and tech workers
% \item Facilitated communication between performers, tech workers, and facility managers
% \end{itemize}

% {\sl Tour Guide} \hfill Sept. 2011 - May 2014 \\
%  Department of Admissions, Macalester College, Saint Paul, MN
% \begin{itemize}
% \item{Honed public speaking and interpersonal skills}
% \end{itemize}

% {\sl Sound \& Stage Technician} \hfill  Sept. 2011 - May 2014 \\
% Department of Music, Macalester College, Saint Paul, MN
% \begin{itemize}
% \item Mastered use of professional grade sound equipment and lighting systems
% \item Communicated and cooperated efficiently in stressful situations
% \end{itemize}

% {\sl Orientation Leader} \hfill Sept. 2012 \\
% Office of Student Affairs, Macalester College, Saint Paul, MN
% \begin{itemize}
% \item{Facilitated a welcoming academic and social environment for students from diverse backgrounds}
% \end{itemize}

% {\sl Nature Program Director} \hfill June - August 2012 \\
% Becket-Chimney Corners YMCA, Becket, MA
% \begin{itemize}
% \item{Led fun and educational nature hikes and activities}
% \item{Designed and carried out daily lesson plans}
% \item{Fostered excitement for science in students}
% \end{itemize}

%----------------------------------------------------------------------------------------
%      PERSONAL SECTION
%----------------------------------------------------------------------------------------

%\section{PERSONAL INTERESTS}

\end{resume}
\end{document}